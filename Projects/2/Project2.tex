\documentclass{article}

\usepackage[tmargin=0.5in,bmargin=0.25in,rmargin=0.1in]{geometry}
\usepackage{amsmath, amssymb, amsthm}
\usepackage{enumitem}
\usepackage{graphicx}
\usepackage{listings}

\graphicspath{.}

\title{\vspace{-5ex}M/CS 375 Project 2}
\author{Isaac Boaz, Cole Yamamura}

\begin{document}
\maketitle

\section*{Problem 1}
\begin{enumerate}[label=(\alph*)]
    \item \(f(x) = 2x^3 - 6x - 1\) can be rewritten as:
        \begin{align*}
            f_1(x) & = \sqrt[3]{3x+\frac{1}{2}} \\
            f_2(x) & = \frac{1}{2x^2-6}  \\
            f_3(x) & = \sqrt[3]{3x+\frac{1}{2}} = f_1
        \end{align*}
    \item Taking the derivatives of these \(f(x)\)'s to show they all converge:
        \begin{align*}
            f_1'(x) &= \sqrt[-2/3]{\frac{1}{2} + 3x}\ \\
                    f_1'(r_1) & = -0.18549\ldots - 0.3212\ldots i \\
                    |f_1'(r_1)| & \approx 0.37099 \\
                    |f_1'(r_1)| & < 1\ \checkmark \\
        \end{align*}
        \begin{align*}
            f_2'(x) &= -\frac{x}{(x^2-3)^2} \\
                    f_2'(r_2) & = 0.0190528213703384 \\
                    |f_2'(r_2)| & = 0.0190528213703384 \\
                    |f_2'(r_2)| & < 1\ \checkmark \\
        \end{align*}
        \begin{align*}
            f_3'(x) &= \sqrt[-2/3]{\frac{1}{2} + 3x} = f_1' \\
            f_3'(r_3) &= 0.06924771700019839 \\
            |f_3'(r_3)|& = 0.06924771700019839 \\
            |f_3'(r_3)| &< 1\ \checkmark
        \end{align*}
\end{enumerate}

\pagebreak

\begin{lstlisting}
 i    xi          g(xi)          error        error/lastError
 0  -2.000000000  -1.765174168   0.358216473
 1  -1.765174168  -1.686340658   0.123390640  0.344458308
 2  -1.686340658  -1.658150305   0.044557131  0.361106244
 3  -1.658150305  -1.647833203   0.016366778  0.367321176
 4  -1.647833203  -1.644024866   0.006049676  0.369631435
 5  -1.644024866  -1.642614632   0.002241339  0.370489116
 6  -1.642614632  -1.642091805   0.000831105  0.370807391
 7  -1.642091805  -1.641897889   0.000308278  0.370925480
 8  -1.641897889  -1.641825954   0.000114362  0.370969292
 9  -1.641825954  -1.641799267   0.000042427  0.370985546
10  -1.641799267  -1.641789367   0.000015740  0.370991576
11  -1.641789367  -1.641785694   0.000005839  0.370993813
12  -1.641785694  -1.641784331   0.000002166  0.370994643
13  -1.641784331  -1.641783826   0.000000804  0.370994951
14  -1.641783826  -1.641783638   0.000000298  0.370995065
15  -1.641783638  -1.641783568   0.000000111  0.370995108
16  -1.641783568  -1.641783543   0.000000041  0.370995122
17  -1.641783543  -1.641783533   0.000000015  0.370995124

ans =

    -1.6418

i    xi          g(xi)          error        error/lastError
0  -1.000000000  -0.250000000   0.831745598
1  -0.250000000  -0.170212766   0.081745598  0.098281973
2  -0.170212766  -0.168291940   0.001958364  0.023956815
3  -0.168291940  -0.168255117   0.000037538  0.019167986
4  -0.168255117  -0.168254415   0.000000715  0.019055470
5  -0.168254415  -0.168254402   0.000000014  0.019076087

ans =

    -0.1683

 i    xi           g(xi)         error        error/lastError
 0   3.000000000   2.117911792   1.189962071
 1   2.117911792   1.899513761   0.307873863  0.258725778
 2   1.899513761   1.836946464   0.089475832  0.290624968
 3   1.836946464   1.818214183   0.026908534  0.300735226
 4   1.818214183   1.812530120   0.008176254  0.303853569
 5   1.812530120   1.810798297   0.002492190  0.304808320
 6   1.810798297   1.810269985   0.000760367  0.305100009
 7   1.810269985   1.810108756   0.000232056  0.305189064
 8   1.810108756   1.810059547   0.000070827  0.305216244
 9   1.810059547   1.810044528   0.000021618  0.305224523
10   1.810044528   1.810039943   0.000006598  0.305226995
11   1.810039943   1.810038544   0.000002014  0.305227568
12   1.810038544   1.810038117   0.000000615  0.305227149
13   1.810038117   1.810037987   0.000000188  0.305225076
14   1.810037987   1.810037947   0.000000057  0.305218069
15   1.810037947   1.810037935   0.000000017  0.305195048

ans =

    1.8100
\end{lstlisting}

\pagebreak

\section*{Problem 2}
\begin{enumerate}[label=(\alph*)]
    \item To find the rate of convergence using Netwon's Method, first find the multiplicity of \(f(x)\) at \(r = 0\)
     \begin{align*}
        f(x) &= e^{sin^3x} + x^6 - 2x^4 - x^3 - 1 \\
        f'(x) &= (6x^3-8x-3)x^2+3e^{sin^3x}sin^2(x)cos(x) \\
        f'(0) &= 0 \\
        f''(x) &= 6x(5x^3-4x-1)-3e^{sin^3(x)}sin^3(x)+3e^{sin^3(x)}(3sin^3(x)+2)sin(x)cos^2(x) \\
        f''(0) &= 0 \\
        f^{(3)}(x) &= 3 \big( 40x^3-16x+e^{sin^3(x)}(9sin^6(x)+18sin^3(x)+2)cos^3(x)-e^{sin^3(x)}sin^2(x)(9sin^3(x)+7)cos(x)-2 \big) \\
        f^{(3)}(0) &= 0 \\
        f^{(4)}(x) &= \cdots \\
        f^{(4)}(0) &= -48 \rightarrow m = 4
    \end{align*}
    Since \(m = 4\), we know that Newton's method would have linear convergence (more precisely, \(e_i \approx \frac{3}{4}e_{i-1}\))
    \item \(f\) has another root in \([1, 2]\) because \(f(1) < 0\) and \(f(2) > 0\).
    \item Plotting \(f\) and looking between \([1, 2]\),  we see that the slope grows exponentially, indicating it has a low multiplicity (ie likely it is a simple root).
\end{enumerate}

\begin{lstlisting}
i 	 xi           error       error/lastError^2
0   2.000000000   0.469866492
1   1.785128336   0.254994828   1.155001163
2   1.638833023   0.108699515   1.671725109
3   1.557786334   0.027652826   2.340368801
4   1.532097594   0.001964086   2.568511006
5   1.530105426   0.000028082   7.279586801
6   1.530134118   0.000000610 773.296975370
7   1.530133495   0.000000013 35433.350581781
8   1.530133508   0.000000000 1639978.902695282

ans =

    1.5301
\end{lstlisting}

\section*{Problem 3}
\begin{enumerate}[label=(\alph*)]
    \item \begin{enumerate}[label=\arabic*.]
        \item Inner for loop is run \((m + 100 - 50) + 1 = m + 51\) times
        \item Outer for loop changes \(m\), so we simply sum \(\sum_{m=1}^{n}\)
    \end{enumerate}
    \begin{equation*}
        \sum_{m=1}^{N}m+51 = \frac{N(N + 103)}{2}
    \end{equation*}
    \item \begin{enumerate}[label=\arabic*.]
        \item Inner for loop is run \((m - 10) + 1 = m - 9\) times
        \item Second loop modifies \(m\), so sum \(\sum_{m=11}^{100}\)
        \item Outmost loop simply runs \(N\) times.
    \end{enumerate}
    \begin{equation*}
        N\cdot\sum_{m=11}^{100}{m-9} = 4185N
    \end{equation*}
    \item \begin{enumerate}[label=\arabic*.]
        \item Inner for loop is run \((m - 1) + 1 = m\) times
        \item Second loop doesn't modify \(m\), so multiply inner by \((m + 1 - 1) + 1 = m + 1\)
        \item Outmost loop modifies \(m\), so sum \(\sum_{m=1}^{N}\)
    \end{enumerate}
    \begin{equation*}
        \sum_{m=1}^{N}{(m+1)m} = \frac{N(N+1)(N+2)}{3}
    \end{equation*}
\end{enumerate}

\end{document}
