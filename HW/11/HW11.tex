\documentclass{article}

\usepackage[tmargin=0.5in,bmargin=0.25in]{geometry}
\usepackage{amsmath, amssymb, amsthm}
\usepackage{enumitem}

\title{\vspace{-5ex}M/CS 375 HW 11}
\author{Isaac Boaz}

\begin{document}
\maketitle

\section*{Problem 8}
If a system of 3000 equations in 3000 unknowns can be solved by Gaussian elimination in 5 seconds on a given computer, how many back substitutions of the same size can be done per second?

\begin{align*}
    \frac{s_b ^2}{2 \cdot s_e^3 / 3} = \frac{t_b}{t_e} \\
    s_e = 3000                                         \\
    t_e = ?                                            \\
    s_b = 3000                                         \\
    t_b = 5                                            \\
    \frac{3000^2}{2 \cdot 3000^3 / 3} = \frac{t_b}{5}  \\
\end{align*}

Simplifying the above, we get \(t_b = \frac{1}{400}\) seconds, meaning it would take 1/400th of a second to do one back substitution. (I.e: we can do 400 back substitutions / second).

\end{document}