\documentclass[10pt]{article}

\usepackage[tmargin=0.1in, bmargin=0.25in, lmargin = 0.1in, rmargin = 0.25in]{geometry}
\usepackage{amsmath, amssymb, amsthm}
\usepackage{enumitem}
\usepackage{multicol}
\usepackage{listings}
\usepackage{pstricks}
\usepackage{tikz}
\newcommand*\circled[1]{\tikz[baseline=(char.base)]{
    \node[shape=circle,draw,inner sep=0.5pt] (char) {#1};}}

\makeatletter
\renewcommand*\env@matrix[1][*\c@MaxMatrixCols c]{%
    \hskip -\arraycolsep
    \let\@ifnextchar\new@ifnextchar
    \array{#1}}
\makeatother

\begin{document}

\begin{multicols}{2}
    \small
    \begin{minipage}{\columnwidth}
        \subsection*{Standard vs. Nested Form}
        \begin{align*}
            \text{Standard Form: } & a_0 + a_1 x + a_2 x^2 + \cdots + a_n x^n  \\
            \text{Nested Form: }   & a_0 + x(a_1 + a_2x + \cdots + a_nx^{n-1})
        \end{align*}
        \subsection*{Sig Fig Rounding}
        \tiny
        4-Digit Example \\
        \small
        \begin{tabular}{rl}
            37.21429      & 37.21    \\
            0.002271828   & 0.002272 \\
            3000527.11059 & 3001000  \\
        \end{tabular}
        \subsection*{Bit Preciscion}
        \begin{tabular}{lcccc}
                             & sign & exponent & mantissa & machine \(\epsilon\) \\
            single           & 1    & 8        & 23       & \(2^{-23}\)          \\
            double (default) & 1    & 11       & 52       & \(2^{-52}\)          \\
            long double      & 1    & 15       & 64       & \(2^{-64}\)
        \end{tabular}
        \begin{multicols}{2}
            \begin{tabular}{|c|c|c|}
                \hline
                sign & exponent & mantissa \\
                \hline
            \end{tabular}
            \\ \\
            \(\rightarrow 1.b_1b_2 \cdots b_n \times 2^P\)
        \end{multicols}
        \subsection*{Error Calculation}
        \begin{align*}
            x_e = \text{Exact}      & ,\ x_a = \text{Approximate}                  \\
            \text{Abs err}          & = \left| x_{e} - x_{a} \right|               \\
            \text{Rel err}          & = \frac{\left| x_{e} - x_{a} \right|}{x_{e}} \\
            \text{Worst case error} & = 0.5 \cdot 10^{-P+1}
        \end{align*}
        \subsection*{Loss of Significance}
        \begin{equation*}
            a - b = \frac{a^2-b^2}{a+b}
        \end{equation*}
        \section*{Theorems}
        \textbf{Intermediate Value Theorem}
        \begin{equation*}
            \begin{aligned}
                 & \text{If } f(x) \text{ is continuous on } [a,b] \text{ and } f(a) \neq f(b) \text{,}              \\
                 & \text{then for any \(y\) in \([f(a),\ f(b)] \) } \exists\ c \in [a,b] \text{ such that } f(c) = y
            \end{aligned}
        \end{equation*}
        \textbf{Mean Value Theorem}
        \begin{equation*}
            \begin{aligned}
                 & \text{If } f'(x) \text{ is continuous on } [a,b] \text{,}                      \\
                 & \text{then } \exists\ c \in [a,b] \text{ such that } f'(c) = (f(b)-f(a))/(b-a)
            \end{aligned}
        \end{equation*}
        \textbf{Rolle's Theorem}
        \begin{equation*}
            \begin{aligned}
                 & \text{If } f'(x) \text{ is continuous on } [a,b] \text{ and } f(a) = f(b) \text{,} \\
                 & \text{then } \exists\ c \in [a,b] \text{ such that } f'(c) = 0
            \end{aligned}
        \end{equation*}
        \subsection*{Taylor Series}
        \begin{align*}
            P_k(x) = \sum_{n=0}^k \frac{f^{(n)}(x_0)}{n!} (x-x_0)^n \\
            E_{k+1} = \frac{f^{(k+1)}(c)}{(k + 1)!}  (x-x_0) ^ {k+1}
        \end{align*}
    \end{minipage}
    \begin{minipage}{\columnwidth}
        \begin{flushright}
            \subsection*{Methods}
            \textbf{Bisection Method}
            \begin{enumerate}
                \item Choose \(a\), \(b\) such that \(f(a) \cdot f(b) < 0\)
                \item Compute \(c = \frac{a+b}{2}\)
                \item If \(f(c)\) and \(f(a)\) have opposite signs, then \(b = c\)
                \item If \(f(c)\) and \(f(b)\) have opposite signs, then \(a = c\)
                \item If \(|b - a| < \epsilon\) or \(f(c) = 0\) then \(c\) is the solution
            \end{enumerate}
            \begin{equation*}
                \begin{aligned}
                    \text{Error} = \frac{b - a}{2^{n + 1}}                                                         & \\
                    \text{Steps(\(\epsilon\))} = \bigl\lceil \frac{\ln(b-a) - \ln(\epsilon)}{\ln(2)}-1 \bigr\rceil &
                \end{aligned}
            \end{equation*}
            \textbf{Newton's Method}
            \begin{enumerate}
                \item \(x_{n+1} = x_n - \frac{f(x_n)}{f'(x_n)}\)
                \item If \(|x_n - x_{n-1}| < \epsilon\) or \(f(x_n) = 0\) then \(x_n\) is the solution
            \end{enumerate}
            Multiplicity
            \begin{equation*}
                \begin{aligned}
                    \text{If } f^{(m-1)}(r_0) = 0 \text{ and } f^{(m)}(r_0) \neq 0 \text{,} & \\
                    \text{then } f^{(m)}(r_0) \text{ is a root of multiplicity } m          &
                \end{aligned}
            \end{equation*}
            Convergence
            \vspace*{-1em}
            \setlength{\columnsep}{-1in}
            \begin{multicols}{2}
                \begin{minipage}{1in}
                    \(m = 1\)
                    \begin{equation*}
                        \begin{aligned}
                             & M = \frac{1}{2} \left| \frac{f''(r_0)}{f'(r_0)} \right| \\
                             & e_{i+1} \approx Me_i^2                                  \\
                        \end{aligned}
                    \end{equation*}
                \end{minipage} \\
                \begin{minipage}{1in}
                    \(m > 1\)
                    \begin{equation*}
                        \begin{aligned}
                            e_{i+1} \approx \frac{m-1}{m} e_i
                        \end{aligned}
                    \end{equation*}
                \end{minipage}
            \end{multicols}
            \textbf{Secant Method}
            \begin{equation*}
                x_{n+1} = x_n - f(x_n)\frac{x_n - x_{n-1}}{f(x_n) - f(x_{n-1})}
            \end{equation*}
            Convergence
            \begin{equation*}
                \begin{aligned}
                    e_{i+1} \approx Me_i^p                            & \\
                    p = \frac{1 + \sqrt{5}}{2} \text{ (Golden Ratio)} &
                \end{aligned}
            \end{equation*}
            \textbf{False Position Method} \\
            Same as Bisection Method, but
            \begin{equation*}
                c = a - \frac{f(a) (b-a)}{f(b) - f(a)}
            \end{equation*}
            Convergence - Generally same as Bisection \\
            \textbf{Fixed Point Iteration}
            \begin{enumerate}
                \item Solve \(f(x)\) for \(x = g(x)\) \\
                      \begin{align*}
                          g(x) = 2.8x - x^2 = x \\
                          1.8x -x^2 = 0         \\
                          x(1.8 - x) = 0        \\
                          x = 0, 1.8
                      \end{align*}
                      \begin{align*}
                          f(x) = 2x^3 - 6x - 1 \\
                          g(x) = \frac{2x^3-1}{6}
                      \end{align*}
                \item If \(|g'(r_n)| < 1\), fixed point will converge to \(r_n\)
            \end{enumerate}
        \end{flushright}
    \end{minipage}
\end{multicols}

\begin{center}
    \rule{4in}{1pt}
\end{center}

\vspace{-1.5em}

\begin{multicols}{3}
    \begin{minipage}{\columnwidth}
        \subsection*{Matrix}
        \begin{description}
            \item[Gaussian Elimination] \(O(\frac{2n^3}{3})\)
            \item[Back Substitution] \(O(n^2)\)
            \item[LU Fact.] Gaussian + 2 * Back Sub
        \end{description}
    \end{minipage}

    \begin{minipage}{\columnwidth}
        \subsection*{Summations}
        % \setlength{\columnsep}{0in}
        \begin{multicols}{2}
            \vspace*{-2em}
            \begin{equation*}
                \begin{aligned}
                     & \sum_{i=1}^{n} i = \frac{n(n+1)}{2}         & \\
                     & \sum_{i=1}^{n} i^2 = \frac{n(n+1)(2n+1)}{6} & \\
                \end{aligned}
            \end{equation*}
            \columnbreak
            \vspace*{-2.0em}
            \begin{equation*}
                \begin{aligned}
                    \sum_{i=1}^{n+5}{(k-1)} = \sum_{i=1}^{n+4} j & = \frac{(n+4)(n+5)}{2} & \\
                \end{aligned}
            \end{equation*}
        \end{multicols}
    \end{minipage}

    \begin{minipage}{1.3\columnwidth}
        \begin{center}
            \subsection*{Complexity Ratio}
        \end{center}
        \begin{equation*}
            \frac{s_b^2}{2\cdot s_e^3 / 3} = \frac{t_b}{t_e}
        \end{equation*}
    \end{minipage}
\end{multicols}

\pagebreak

\begin{multicols}{2}
    \begin{minipage}{\columnwidth}
        \subsection*{LU Factorization}
        Goal: \(A = LU \rightarrow A\overrightarrow{x} = \overrightarrow{b} \) \\
        \textbf{Slow Method}
        \begin{enumerate}
            \item Row reduce \(A\), recording operations as elementary matrices
            \item The row reduced \(A\) is now \(U\)
            \item You now have \(E_n E_{n-1} \cdots E_2 E_1 A = LU\)
            \item Invert the elementary matricies to find L
            \item \(L = (E_n E_{n-1} \cdots E_2 E_1) ^ {-1}\)
        \end{enumerate}

        Example
        \begin{minipage}{0.5\linewidth}
            \[
                A = \begin{bmatrix}
                    1  & 2 & -1 \\
                    2  & 1 & -2 \\
                    -3 & 1 & 1
                \end{bmatrix}
            \]
            \[
                \begin{bmatrix}
                    1 & 0            & 0 \\
                    0 & 1            & 0 \\
                    0 & -\frac{7}{3} & 1
                \end{bmatrix}
                \begin{bmatrix}
                    1 & 0 & 0 \\
                    0 & 1 & 0 \\
                    3 & 0 & 1
                \end{bmatrix}
                \begin{bmatrix}
                    1  & 0 & 0 \\
                    -2 & 1 & 0 \\
                    0  & 0 & 1
                \end{bmatrix}
            \]
            \begin{center}
                \(e_3 \quad\quad\quad\quad e_2 \quad\quad\quad\quad e_1\)
            \end{center}

            \begin{align*}
                E_3 E_2 E_1 A = U                \\
                A = E_1^{-1} E_2^{-1} E_3^{-1} U \\
                = (E_3 E_2 E_1)^{-1} U           \\
                A = LU = \begin{bmatrix}
                             1  & 0            & 0 \\
                             2  & 1            & 0 \\
                             -3 & -\frac{7}{3} & 1
                         \end{bmatrix}
                \begin{bmatrix}
                    1 & 2  & -1 \\
                    0 & -3 & 0  \\
                    0 & 0  & -2
                \end{bmatrix}
            \end{align*}
        \end{minipage} \\
        \textbf{Fast Method}
        \begin{enumerate}
            \item Apply pivots internally within \(A\)
            \item Note that you will subtract rather than add with these operations
        \end{enumerate}

        \begin{minipage}{0.5\linewidth}
            \begin{align}
                \begin{bmatrix}
                    1  & 2 & -1 \\
                    2  & 1 & -2 \\
                    -3 & 1 & 1
                \end{bmatrix} \\
                \begin{bmatrix}
                    1                          & 2  & -1 \\
                    \circled{\(\frac{2}{1}\)}  & -3 & 0  \\
                    \circled{\(-\frac{3}{1}\)} & -7 & -2
                \end{bmatrix}
            \end{align}
        \end{minipage}
        \begin{enumerate}[resume]
            \item Don't forget to recurse through the inner matrix (row 2 and beyond)
        \end{enumerate}
        \begin{minipage}{0.5\linewidth}
            \begin{align}
                \begin{bmatrix}
                    1                          & 2                         & -1 \\
                    \circled{\(\frac{2}{1}\)}  & -3                        & 0  \\
                    \circled{\(-\frac{3}{1}\)} & \circled{\(\frac{7}{3}\)} & -2
                \end{bmatrix}
            \end{align}
        \end{minipage}
        \begin{enumerate}[resume]
            \item The circled (lower triangular) elements are now \(L\) (plus the identity matrix)
            \item The remaining upper triangular elements are \(U\)
        \end{enumerate}
        \subsection*{Derivatives}
        \small
        \begin{equation*}
            \begin{aligned}
                \frac{df}{dx} f(g(x)) = f'(g(x))g'(x) & \quad \frac{d}{dx} ax^n = nax^{n-1} & \frac{d}{dx} ln_a(x) = \frac{1}{xln(a)} \\
                \frac{d}{dx} a^x = \ln(a) a^x         & \quad \frac{d}{dx} sin(x) = cos(x)  & \frac{d}{dx} cos(x) = sin(x)            \\
            \end{aligned}
        \end{equation*}
    \end{minipage}

    \begin{minipage}{0.95\columnwidth}
        \subsection*{PA = LU Factorization}
        Goal: \(PA = LU \rightarrow A\overrightarrow{x} = \overrightarrow{b}\) \\
        \begin{enumerate}
            \item Same logic as LU, but move the \(|\)biggest\(|\) element to the top of the column
            \item Make sure you do the same for inner matrices
            \item You'll have multiple \(P\) matrices, such that
                  \begin{equation*}
                      PA = LU
                  \end{equation*}
            \item Make sure you apply \(P\) when solving for \(\overrightarrow{x}\)
        \end{enumerate}

        \subsection*{Solving}
        \begin{enumerate}
            \item Solve \(L\overrightarrow{y} = \overrightarrow{b}\) for \(\overrightarrow{y}\)
            \item Solve \(U\overrightarrow{x} = \overrightarrow{y}\) for \(\overrightarrow{x}\)
        \end{enumerate}
        \vspace{1em}
        Example 1: \(A = LU\)
        \begin{equation*}
            \text{Let } \overrightarrow{b} = \begin{bmatrix}
                2 \\
                1 \\
                -1
            \end{bmatrix}
        \end{equation*}

        \begin{enumerate}
            \item \(L\overrightarrow{y} = \overrightarrow{b}\) \begin{equation*}
                      \begin{bmatrix}[ccc|c]
                          1  & 0            & 0 & 2  \\
                          2  & 1            & 0 & 1  \\
                          -3 & -\frac{7}{3} & 1 & -1
                      \end{bmatrix}
                      \rightarrow
                      \overrightarrow{y} = \begin{pmatrix}
                          2  \\
                          -3 \\
                          -2
                      \end{pmatrix}
                  \end{equation*}
            \item \(U \overrightarrow{x} = \overrightarrow{y}\) \begin{equation*}
                      \begin{bmatrix}[ccc|c]
                          1 & 2  & -1 & 2  \\
                          0 & -3 & 0  & -3 \\
                          0 & 0  & -2 & -2
                      \end{bmatrix}
                      \rightarrow \overrightarrow{x} = \begin{pmatrix}
                          1 \\
                          1 \\
                          1
                      \end{pmatrix}
                  \end{equation*}
        \end{enumerate}

        Example 2: - \(PA = LU\)
        \begin{equation*}
            \text{Let } \overrightarrow{b} = \begin{bmatrix}
                1 \\
                2 \\
                3
            \end{bmatrix}
            \text{ and } A = \begin{bmatrix}
                2 & 1 & 5  \\
                4 & 4 & -4 \\
                1 & 3 & 1
            \end{bmatrix}
        \end{equation*}
        \begin{enumerate}
            \item Find \(P\) such that \(A\)'s pivots are the largest in magnitude per column. \\
                  \footnotesize In this instance:
                  \begin{equation*}
                      P = P_2 P_1
                      \begin{bmatrix}
                          1 & 0 & 0 \\
                          0 & 0 & 1 \\
                          0 & 1 & 0
                      \end{bmatrix}
                      \begin{bmatrix}
                          0 & 1 & 0 \\
                          1 & 0 & 0 \\
                          0 & 0 & 1
                      \end{bmatrix}
                      = \begin{bmatrix}
                          0 & 1 & 0 \\
                          0 & 0 & 1 \\
                          1 & 0 & 0
                      \end{bmatrix}
                  \end{equation*}
                  \normalsize
            \item \(L \overrightarrow{y} = P\overrightarrow{b}\)
                  \begin{equation*}
                      \begin{bmatrix}[ccc|c]
                          1           & 0            & 0 & 0 \\
                          \frac{1}{4} & 1            & 0 & 6 \\
                          \frac{1}{2} & -\frac{1}{2} & 1 & 5
                      \end{bmatrix}
                      \rightarrow \overrightarrow{y} = \begin{pmatrix}
                          0 \\
                          6 \\
                          8
                      \end{pmatrix}
                  \end{equation*}
            \item \(U \overrightarrow{x} = \overrightarrow{y}\)
                  \begin{equation*}
                      \begin{bmatrix}[ccc|c]
                          4 & 4 & -4 & 0 \\
                          0 & 2 & 2  & 6 \\
                          0 & 0 & 8  & 8
                      \end{bmatrix}
                      \rightarrow \overrightarrow{x} = \begin{pmatrix}
                          -1 \\
                          2  \\
                          1
                      \end{pmatrix}
                  \end{equation*}
        \end{enumerate}
    \end{minipage}
\end{multicols}

\end{document}